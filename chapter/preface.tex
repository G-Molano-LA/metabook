% !TEX root = ../notes_template.tex
\chapter*{Preface}
\addcontentsline{toc}{chapter}{Preface}
\minitoc

% \lipsum % dummy text - remove from real document

\section{Features of this template}
% \epigraph{\emph{... nature isn't classical, dammit, and if you want to make a simulation of nature, you'd better make it quantum mechanical, and by golly it's a wonderful problem, because it doesn't look so easy.}}{Richard Feynman (1981) Simulating physics with computers}
\epigraph{\emph{TeX, stylized within the system as \LaTeX, is a typesetting system which was designed and written by Donald Knuth and first released in 1978. TeX is a popular means of typesetting complex mathematical formulae; it has been noted as one of the most sophisticated digital typographical systems.}}{- \href{https://en.wikipedia.org/wiki/TeX}{Wikipedia}}

\subsection{crossref}
different styles of clickable definitions and theorems
\begin{itemize}
	\item nameref:
		\nameref{def:gaussian_distribution}

	\item autoref:
		\autoref{def:gaussian_distribution}

	\item cref:
		\cref{def:gaussian_distribution},

	\item hyperref:
		\hyperref[def:gaussian_distribution]{Gaussian},
\end{itemize}

\subsection{ToC (Table of Content)}
\begin{itemize}
	\item mini toc of sections at the beginning of each chapter
	\item list of theorems, definitions, figures
	\item the chapter titles are bi-directional linked
\end{itemize}

\subsection{header and footer}
fancyhdr
\begin{itemize}
	\item right header: section name and link to the beginning of the section
	\item left header: chapter title and link to the beginning of the chapter
	\item footer: page number linked to ToC of the whole document
\end{itemize}

\subsection{bib}
\begin{itemize}
	\item titles of reference is linked to the publisher webpage e.g., \cite{kitaev2002classical}
	\item backref (go to the page where the reference is cited) e.g., \cite{childsUniversalComputationQuantum2009}
	\item customized video entry in reference like in \cite{babaiGraphIsomorphismQuasipolynomial2016}
\end{itemize}

\subsection{preface, index, quote (epigraph) and appendix}
\myindex{index} page at the end of this document...

\subsection{symbol and glossary (abbreviation)}
examples: 
\gls{real_number},
% \gls{natural_number},
% \gls{complex_number},
\gls{svm},
\gls{v}

\subsubsection{usage}
\begin{itemize}
	\item glossary package 
	\begin{verbatim}
		pdflatex notes_template.tex
		makeglossaries notes_template
		pdflatex notes_template.tex	
	\end{verbatim}

	\item glossary-extra package and bib2gls
	\begin{verbatim}
		pdflatex notes_template.tex
		bib2gls notes_template
		pdflatex notes_template.tex	
	\end{verbatim}
\end{itemize}

