% !TEX root = ../notes_template.tex
\chapter*{Preface}
\addcontentsline{toc}{chapter}{Preface}
\minitoc

% \lipsum % dummy text - remove from real document

\section{Features of this template}
% \epigraph{\emph{... nature isn't classical, dammit, and if you want to make a simulation of nature, you'd better make it quantum mechanical, and by golly it's a wonderful problem, because it doesn't look so easy.}}{Richard Feynman (1981) Simulating physics with computers}
\epigraph{\emph{TeX, stylized within the system as \LaTeX, is a typesetting system which was designed and written by Donald Knuth and first released in 1978. TeX is a popular means of typesetting complex mathematical formulae; it has been noted as one of the most sophisticated digital typographical systems.}}{- \href{https://en.wikipedia.org/wiki/TeX}{Wikipedia}}

\subsection{crossref}
different styles of clickable definitions and theorems
\begin{itemize}
	\item nameref:
		\nameref{def:gaussian_distribution}

	\item autoref:
		\autoref{def:gaussian_distribution},
		\autoref{alg:miller_rabin}

	\item cref:
		\cref{def:gaussian_distribution},

	\item hyperref:
		\hyperref[def:gaussian_distribution]{Gaussian},
\end{itemize}

\subsection{ToC (Table of Content)}
\begin{itemize}
	\item mini toc of sections at the beginning of each chapter
	\item list of theorems, definitions, figures
	\item the chapter titles are bi-directional linked
\end{itemize}

\subsection{header and footer}
fancyhdr
\begin{itemize}
	\item right header: section name and link to the beginning of the section
	\item left header: chapter title and link to the beginning of the chapter
	\item footer: page number linked to ToC of the whole document
\end{itemize}

\subsection{bib}
\begin{itemize}
	\item titles of reference is linked to the publisher webpage e.g., \cite{kitaev2002classical}
	\item backref (go to the page where the reference is cited) e.g., \cite{childsUniversalComputationQuantum2009}
	\item customized video entry in reference like in \cite{babaiGraphIsomorphismQuasipolynomial2016}
\end{itemize}

\subsection{preface, index, quote (epigraph) and appendix}
\myindex{index} page at the end of this document...

\subsection{symbol and glossary (abbreviation)}
examples: 
\gls{real_number},
% \gls{natural_number},
% \gls{complex_number},
\gls{svm},
\gls{v}

\subsubsection{usage}
\begin{itemize}
	\item glossary package 
	\begin{verbatim}
		pdflatex notes_template.tex
		makeglossaries notes_template
		pdflatex notes_template.tex	
	\end{verbatim}

	\item glossary-extra package and bib2gls
	\begin{verbatim}
		pdflatex notes_template.tex
		bib2gls notes_template
		pdflatex notes_template.tex	
	\end{verbatim}
\end{itemize}

\section{Related Tools}
\subsection{VSCode}
Extension: \href{https://marketplace.visualstudio.com/items?itemName=James-Yu.latex-workshop}{Latex Workshop by James Yu}

\subsubsection{settings}

\subsection{lualatex and latexmk}
.latexmkrc configuration file
\begin{verbatim*}
	$pdflatex = 'lualatex -synctex=1 -interaction=nonstopmode --shell-escape %O %S';
	@generated_exts = (@generated_exts, 'synctex.gz');
	$pdf_mode = 1;

	add_cus_dep('glo', 'gls', 0, 'makeglo2gls');
	sub makeglo2gls {
		system("makeindex -s '$_[0]'.ist -t '$_[0]'.glg -o '$_[0]'.gls '$_[0]'.glo");
	}
\end{verbatim*}
To explain ....
\begin{verbatim}
# Also delete the *.glstex files from package glossaries-extra. Problem is,
# that that package generates files of the form "basename-digit.glstex" if
# multiple glossaries are present. Latexmk looks for "basename.glstex" and so
# does not find those. For that purpose, use wildcard.
$clean_ext = "%R-*.glstex";

push @generated_exts, 'glstex', 'glg';

add_cus_dep('aux', 'glstex', 0, 'run_bib2gls');

# PERL subroutine. $_[0] is the argument (filename in this case).
# File from author from here: https://tex.stackexchange.com/a/401979/120853
sub run_bib2gls {
    if ( $silent ) {
    #    my $ret = system "bib2gls --silent --group '$_[0]'"; # Original version
        my $ret = system "bib2gls --silent --group $_[0]"; # Runs in PowerShell
    } else {
    #    my $ret = system "bib2gls --group '$_[0]'"; # Original version
        my $ret = system "bib2gls --group $_[0]"; # Runs in PowerShell
    };

    my ($base, $path) = fileparse( $_[0] );
    if ($path && -e "$base.glstex") {
        rename "$base.glstex", "$path$base.glstex";
    }

    # Analyze log file.
    local *LOG;
    $LOG = "$_[0].glg";
    if (!$ret && -e $LOG) {
        open LOG, "<$LOG";
    while (<LOG>) {
            if (/^Reading (.*\.bib)\s$/) {
        rdb_ensure_file( $rule, $1 );
        }
    }
    close LOG;
    }
    return $ret;
}
\end{verbatim}

\section{Copyright and License}

\begin{itemize}
    \item GitHub Repo: \url{https://github.com/Jue-Xu/Latex-Template-for-Scientific-Style-Book}
    \item Overleaf template: \url{https://www.overleaf.com/latex/templates/latex-template-for-scientific-style-book/ntprxjksmqxx}
\end{itemize}


