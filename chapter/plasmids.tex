% !TEX root = ../notes_template.tex
\chapter{Plasmids}\label{chp:Plasmids}

\minitoc
 

\section{Introduction}
Plasmid are DNA molecules located outside of the chromosomal DNA, i.e. extrachromosomal. Their topology is frequently 
circular although linear plasmid also exists. They have been extensively studied in Bacteria, even though Archaea and 
Eukaryota also carry them. 

Plasmids are generally associated with a host range, which can be broad or narrow. Incompatibility groups used from plasmid 
classification. Seems that their host range is determined by their ability to escape host defenses and the use of host's machinery. 

Mainly, plasmids mobility has been associated to its conjugation system, although other mechanisms have been described, 
such as membrane vesicles (castañeda 2024). Importantly, not all plasmids have transfer and mobility functions. 

Due to their mobility, plasmids are included in the category of mobile genetic elements, along with 
Integrative Conjugative Elements (ICEs, which integrate into the host genome and carry a functional conjugation system for 
inter-cellular transfer, \(\backsim\)18-500 kb in length) and phages (forming viral particles that infect a prokaryotic cell, 
replicating within it and are transferred between the cells via transduction, \(\backsim\)11-500 kb in length) (Khedkar 2022). 

Insertion sequences (IS, elements carry only a transposase gene, \(\backsim\)2.5 kb in length) and, transposons (elements 
that carry transposase and dispensable cargo genes, \(\backsim\)5 kb in length) and integrons (gene acquisition systems 
that are immobile without other MGEs, several kb in length) depend on other MGEs for inter-cellular transfer.

\section{Structure}
\begin{itemize}
    \item \textbf{Backbone}. Consists in two differentiated parts.
    \begin{itemize}
        \item Essential genes that ensure vertical inheritance (replication, copy number, partitioning, stability).
        \item Inessential genes that code for horizontal gene transfers.
    \end{itemize}
    \item \textbf{Genetic cargo}. Regions outside backbone that may contribute a phenotypic advantage to their hosts. 
\end{itemize}
Plasmid stability depends on the balance between genetic burden and beneficial effect to the host (genetic cargo).

\section{Replication}
The fundamental characteristic that defines a plasmid is its ability to replicate autonomously. This independence allows 
the plasmid to present a copy number higher than the chromosome. 
\begin{align*}
    Plasmid\ copy\ number = \frac{\#\ plasmid}{\#\ chromosomal\ copies} 
\end{align*}

However, they also seem to replicate in step with the chromosome, doubling in number during the cell growth of their host, 
being vertically inherited from generation to generation. Plasmids use at least three distinct types of replication systems: 
rolling circle, theta, and linear replication. \textbf{Rolling circle} is generally confined to small and high copy number 
plasmids, whereas large and low copy number plasmids invariably use types of \textbf{theta} or \text{linear} replication systems. 

Plasmids are replicons that transfer between cells via conjugation (6), up to 2.5 Mb in length. This independence from the 
chromosome defines them as genetic locus where genes may evolve faster than in the chromosome.

\section{Toxin-Antitoxin systems}
Many bacteria encode lethal proteins in their genome alongside antidotes that counteract their toxicity. These toxin-antitoxin (TA) 
systems are classified into different types according to the nature of the antitoxins and the mechanism of action of the toxins.

\subsection{Hok/sok system}
The hok/Sok system has been the most studied T1TA (RNA/RNA interacting systems). It was first discovered on 
\textit{Escherichia coli} R1 plasmid where it acts by maintaining plasmid copies in a cell population through post-segregational 
killing of the plasmid-free cells. 

\begin{itemize}
    \item The Hok (host-killing) type I toxin is a small hydrophobic protein [52 amino acids (aa)] targeting the inner membrane and 
        leading to cell death.
    \item The Sok (suppression of killing) antitoxin is an RNA that inhibits the production of Hok at the post-transcriptional level.
    \item The mok (modulation of killing), that overlaps with the hok coding sequence (CDS) and is required for hok translation. 
    The translation of mok, rather than the Mok product, was shown to be important for proper hok regulation and expression. 
    For simplicity, the mok\_hok bicistronic mRNA will be referred to as the hok mRNA throughout the article \citetitle{Rhun2022} \cite{Rhun2022}.
\end{itemize}

[TO DO]
\begin{itemize}
    \item \citetitle{Khedkar2022} \cite{Khedkar2022}
    \item \citetitle{Harms2018} \cite{Harms2018}
\end{itemize}
