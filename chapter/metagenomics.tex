% !TEX root = ../notes_template.tex
\chapter{Metagenomics}\label{chp:metagenomics}

\minitoc

\section{Metagenomes reconstruction}
 
Binning. Critical step required to establish a genome from a metagenomic assembly. This involves assignment of assembled 
fragments to a draft genome based on detection on any scaffold of some signal(s) that occur(s) locally within a genome and 
persists genome-wide \cite{Chen2020}. 

Genome curation \cite{Hiltemann2023,microbiome-metagenomics-binning}. Filling scaffolding gaps and removal of local assembly 
errors. Gap filling strategies: 
\begin{itemize}
    \item[GapFiller] Tool for filling the N's gaps at scaffold joins. Often a few iterations are needed for gap closure. Using:
    \begin{itemize}
        \item Unplaced pairs for reads adjacent to the gaps. When reads are mapped to genome fragments that compose a bin, a 
        file of unplaced paired reads is generated for each fragment.
        \begin{itemize}
            \item If due to low coverage gap filling is not achieve, potentially use of reads from other sample in which the sample population occurs.
            \item Deeper sequencing of the same sample.
        \end{itemize}
        \item Placement of full metagenomic read data set to the new version of the scaffold.
        \item Use of misplaced reads. This can be useful in cases where the necessary reads are misplaced, either elsewhere 
        on that scaffold or on another scaffold in the bin. Misplaced read identification:
        \begin{itemize}
            \item Read pileups with anomalously high frequencies of SNVs in a subset of reads.
            \item Read pairs point outward (rather than toward each other, as expected).
            \item Unusually long paired read distances.
        \end{itemize}
        \item Sometimes even with sufficient read depth, gap filling cannot be achieved due to complex repeats. 
        Sometimes these repeat regions can be resolved careful read-by-read analysis, often requiring relocation of reads 
        based on the placement of their pairs and sequence identity.
    \end{itemize}
\end{itemize}

Local assembly errors (from more common to less):
\begin{itemize}
    \item Error I:
    \begin{description}
        \item[Identification] Sequence in that region lacks perfect support, by even one read.
        \item[Solution] Consensus sequence should be replaced by Ns (gap), which can be further filled.
        \item[Example] \url{https://genome.cshlp.org/content/suppl/2020/03/18/gr.258640.119.DC1/Supplemental_Fig_S3.pdf}.
    \end{description}  
    \item Error II:
    \begin{description}
        \item[Identification] Ns have been inserted during scaffolding despite overlap between the flanking sequences.
        \item[Solution] Close the gap, eliminating the Ns and the duplicated sequence.
        \item[Example] \url{https://genome.cshlp.org/content/suppl/2020/03/18/gr.258640.119.DC1/Supplemental_Fig_S4.pdf}.
    \end{description}
    \item Error III:
    \begin{description}
        \item[Identification] Incorrect number of repeats has been incorporated into the scaffold sequences. Anomalous read 
        depth over that region.
    \end{description}
    \item Error IV: Chimera sequences from two different organisms.
    \begin{description}
        \item[Identification] These joints typically lack paired read support and/or can be identified by very different 
        coverage values and/or phylogenetic profiles on either side of the join.
    \end{description}
    \item Error V: Artificial concatenation of an identical sequence.
    \begin{description}
        \item[Identification] Repeat finder.
    \end{description}
\end{itemize}

\section{Pangenomes}
[TO DO]
Differents genes within a population. Pangenome analysis.

\section{Microbila diversity}
[TO DO]
Beta diversity and its representation. Concept of dimensionality reduction techniques.

\section{Abundance estimation}
[TO DO]
How to quantify composition: markers genes and what makes good a marker gene (to be single and present in the core).

\section{Sequencing depth}
[TO DO]
Huttenhower -> For strain analysis = ideally 10X; Gene-absence: \(\backsim\)1X
Discoveries and findings with microbelix. 