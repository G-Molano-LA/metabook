% !TEX root = ./notes_template.tex
% \usepackage[style=super]{glossaries}
% https://www.overleaf.com/learn/latex/Glossaries
\usepackage[style=super,toc,acronym]{glossaries}
\setlength{\glsdescwidth}{1\linewidth}

\makeglossaries

\renewcommand\glossaryname{List of Abbreviations and Symbols}

\newglossaryentry{latex}
{
        name=latex,
        description={Is a mark up language specially suited for 
scientific documents}
}

\newglossaryentry{bulge}{
        name=bulge,
        description={Two short alternative directed paths between the same vertices of the de Bruijn graph. They are 
        originated from sequencing errors, minor variations between sequences, or repetitive regions}
}
\newglossaryentry{complex bulge}{
        name=complex bulge,
        description={Multiple bulges often aggregate into more complex subgraphs}
}

\newacronym{crp}{CRP}{C-reactive protein}
\newacronym{t2dm}{T2DM}{Type-2 Diabetes Mellitus}
\newacronym{cvd}{CVD}{cardiovascular disease}