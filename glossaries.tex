% !TEX root = ./notes_template.tex
\usepackage[style=super,toc,acronym]{glossaries}
\setlength{\glsdescwidth}{1\linewidth}

\makeglossaries

\renewcommand\glossaryname{List of Abbreviations and Symbols}

%%
% GLOSSARY
%%
\newglossaryentry{bulge}{
        name=bulge,
        description={Two short alternative directed paths between the same vertices of the de Bruijn graph. They are 
        originated from sequencing errors, minor variations between sequences, or repetitive regions}
}
\newglossaryentry{complex bulge}{
        name=complex bulge,
        description={Multiple bulges often aggregate into more complex subgraphs}
}
\newglossaryentry{ecosystems}{
        name=ecosystem,
        description={Community or group of organisms that live and interact with each other in a specific environment}
}

\newglossaryentry{microbiome}{
        name=microbiome,
        description={The words "micro" and "biome" are of Ancient Greek origin. "Micro" ($\mu\iota\kappa\rho o\zeta$) means small, while 
        the term "biome" is composed of the Greek word bíos ($\beta\iota o \zeta$, life) and modified by the ending "ome" (Anglicization of Greek)}
}

\newglossaryentry{microbiota}{
        name=microbiota,
        description={The words "micro" and “biota” are also of Ancient Greek origin. It is a combination of "Micro" ($\mu\iota\kappa\rho o \zeta$, small), with 
        the term "biota" ($\beta\iota o \tau\alpha$), which means the living organisms of an ecosystem or a particular area}
}
\newglossaryentry{habitat}{
        name=habitat,
        description={Refers to externalities such as the physical space and chemical environment that allow and organism to exist, 
        including contributions from other members of the microbial community}
}
\newglossaryentry{niche}{
        name=niche,
        description={Refers to the activity of an organism and the functional role that each member plays in the community. Interactions of 
        the member both with one another and with the habitat drive the emergent organization of the community as a whole}
}
\newglossaryentry{low-grade systemic inflammation}{
        name=low-grade systemic inflammation,
        description={C-reactive protein levels 3-10 mg/L}
}
%%
% ACRONYM
%%

\newacronym{crp}{CRP}{C-reactive protein}
\newacronym{t2dm}{T2DM}{Type-2 Diabetes Mellitus}
\newacronym{cvd}{CVD}{cardiovascular disease}

